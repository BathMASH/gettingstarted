\documentclass[a4paper,12pt]{extarticle}
\usepackage[a4paper,margin=2cm]{geometry}
\usepackage{parskip}
\usepackage{hyperref}
\usepackage{graphicx}
\usepackage{amsmath,amssymb}
\usepackage[mathscr]{eucal}
\newcommand{\abs}{\operatorname{abs}}

\title{Getting started with \LaTeX}
\author{Emma Cliffe}
\date{}
\begin{document}
%\renewcommand{\familydefault}{phv}
%\fontfamily{phv}\selectfont
\maketitle
\tableofcontents

\section{Structuring content}

\subsection{Paragraphs}

This is a first document!

%We can write a comment which does not appear in the output
We leave white space to create a new paragraph.

\subsection{Changing the font}

We can increase the font size and change the font.

\subsection{Emphasis}

We can emphasize the text using \textbf{bold}, \textit{italics} or \texttt{typewriter font}.

\subsection{Verbatim}

To quote computer code or output from mathematical software use the \verb=\verb= command and verbatim environment:
\begin{verbatim}
Layout is
  preserved.
\end{verbatim}

\subsection{Lists}

\subsubsection{Bullet point}

\begin{itemize}
\item Point 1
\item Point 2
\begin{itemize}
\item Sublist
\end{itemize}
\end{itemize}

\subsubsection{Numbered}

\begin{enumerate}
\item Point 1
\begin{enumerate}
\item Sublist
\end{enumerate}
\item A final point
\end{enumerate}

\subsection{Tables}

\begin{tabular}{|p{0.45\textwidth}|p{0.45\textwidth}|}
\hline
We can put \begin{itemize}
\item maths, \item paragraphs, \item lists etc. 
\end{itemize} in this kind of table --- this is not always the case. 
& \bigskip More complex behaviour and layout is beyond the scope of this tutorial \\
\hline
\begin{verbatim}
Some verbatim text
\end{verbatim}
& \begin{center}
This will enable you to create basic tables
\end{center}
\\
\hline
\end{tabular}

\subsection{Images}

This method will work for PDF, JPG and PNG files.

\includegraphics[width=0.8\textwidth]{sigma-logo.png}

\subsection{Citations and bibliography}

To reference we make a citation \cite{latex,ctan} and add the citation information to the bibliography.

\section{Mathematics}

Equations may be within text: \(x\) or displayed on a line alone: \[x+y.\]

\subsection{Basic structures and symbols}

Alphabetic and numerical keyboard characters can be input into equations, also
\[. , ? ! + - = < > ( ) [ ] |\]

Precede percent symbol, dollar and curly braces with a backslash: \( \% \$ \{ \}\).

\subsubsection{Multiplication symbols}

There are two multiplication symbols you might need:
\[2\times 3,\ 2\cdot 3\]

\subsubsection{Superscripts and subscripts}

Superscripts use caret and subscripts use underscore:
\[
a^{b+c}, \quad (a+b)^{(c+d)}, \qquad a_2, \quad (a+b)_{(i+j)}
\]

\subsubsection{Fractions}

Fractions use the frac structure:
\[\frac{1}{2}\]

\subsubsection{Roots}

Roots use the sqrt structure:
\[
\sqrt{2}, \quad \sqrt[n]{2}
\]

\subsubsection{Accents}

You can add accents such as a bar:
\[\bar{x}\]

\subsubsection{Examples: Building up equations}

\[
\bar{x}=\frac{x_1+x_2+x_3+x_4+x_5}{5}
\]

\[
x = \frac{-b\pm \sqrt{b^2-4ac}}{2a}
\]

\subsection{A wide range of symbols and functions}

\subsubsection{Symbols}

Type backslash followed by the symbol name e.g.
\[
x \leq y
\]

\subsubsection{Functions}

Type backslash followed by the function name e.g.
\[
\sin x, \ \sin^{-1} x
\]

Not all function names are available. You can create your own...
\[
\abs x
\]

\subsubsection{Greek}

Type backslash followed by the symbol name e.g. 
\[
\delta,\ \Delta
\]

\subsubsection{Mathematical fonts}

Type backslash followed by the name then the letter in braces (upper case only):
\[
\mathcal{R},\ \mathscr{R},\ \mathfrak{R},\ \mathbb{R}
\]

\subsection{Examples: Standard differentiation notation}

Fraction structure is used: \(\frac{df}{dx}\).This automatically enlarges in a display equation:
\[
\frac{d}{d\theta}(\sin^{-1}\theta)=\frac{1}{\sqrt{1-\theta^2}}
\]

\subsubsection{Standard partial differentiation notation}

Use the \(\partial\) symbol: \(\frac{\partial f}{\partial x}\) e.g.
\[\frac{\partial}{\partial\theta}(f(x,y))\]

\subsection{Multi-sized symbols}

Some symbols exist in multiple sizes and may also have superscripts and subscripts e.g. \(\sum, \ \prod, \ \int\)
\[
\sum, \ \prod, \ \int
\]

\subsubsection{Example: Building up equations with multi-sized symbols}

\[\sum_{k=1}^n = \frac{1}{2}n(n+1)\]

\[\bar{x} = \frac{\sum_{i=1}^n x_i}{n}\]

\[\int_a^b f(u)\frac{du}{dx}dx = \int_{u(a)}^{u(b)}f(u)\, du\]

\subsection{Matrices and piecewise function notation}

Use stretchy brackets, the array structure, a separator and new row symbol:
\[
\left(
\begin{array}{ccc}
1 & 0 & 0 \\
0 & 1 & 0 \\
0 & 0 & 1
\end{array}
\right)
\]

We can use dots:
\[
\left(
\begin{array}{ccc}
1 & \cdots & 0 \\
\vdots & \ddots & \vdots \\
0 & \cdots & 1
\end{array}
\right)
\]

If you only need one of the brackets then match with 'dot':
\[
f(x) = \left\{\begin{array}{rl}
-x, \ &x < 0 \\
x, \ &x \geq 0
\end{array}
\right.
\]

\subsection{Symbol-stack and limits}

To create a new symbol from two others they can be stacked e.g.
\[
\stackrel{x}{\longrightarrow}
\]

\subsubsection{Example: Limit notation}

Limits are \emph{not} build using stackrel e.g. 
\[
e^x = \lim_{n\rightarrow \infty} \left(1 + \frac{x}{n}\right)^n
\]

\subsection{Numbered equations}

To number and refer to a single equation use the equation environment:
\begin{equation}
\label{sum1}
\sum_{k=1}^n k = \frac{1}{2}n(n+1)
\end{equation}

We can refer to the equation \eqref{sum1}.

\subsection{Aligned equations}

To align equations use the align environment, alignment points and new-lines:
\begin{align}
\sum_{k=1}^5 k &= 1 + 2 + 3 + 4 + 5 = 15\nonumber\\
\label{sum2}
&=\frac{1}{2}\cdot 5 \cdot (5+1)
\end{align}

We can refer to any labeled and numbered line in the align environment: \eqref{sum2}.

The align* environment works the same way but will not number any line.

\begin{thebibliography}{99}
\bibitem{latex}
LaTeX Project, \emph{\LaTeX\ --- A document preparation system},
online at \url{http://www.latex-project.org/}

\bibitem{ctan}
CTAN, \emph{Comprehensive TeX Archive Network}, online at \url{http://www.ctan.org/}
\end{thebibliography}

\end{document}
















